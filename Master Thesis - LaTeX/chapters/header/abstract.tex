\begin{abstractgr}%
  Στην παρούσα διπλωματική εργασία εξετάζουμε το πρόβλημα δυναμικής κάλυψης συνόλου ελάχιστου αθροίσματος ($\DSSC$) μια φυσιολογική και ενδιαφέρουσα γενίκευση του κλασικού προβλήματος ανανέωσης λίστας. Στο πρόβλημα $\DSSC$ πρέπει να διατηρήσουμε μια ακολουθία μεταθέσεων $(\pi^0, \pi^1, \ldots, \pi^T)$ n τω πλήθος στοιχείων βάσει μιας ακολουθίας συνόλων κάλυψης $\cR = (R^1, \ldots, R^T)$. Στόχος μας είναι να ελαχιστοποιήσουμε το κόστος ανανέωσης απ' την μετάθεση $\pi^{t-1}$ στην $\pi^{t}$, το οποίο ποσοτικοποιείται με την απόσταση Kendall Tau $\dkt(\pi^{t-1}, \pi^t)$, συν το συνολικό κόστος κάλυψης κάθε αιτήματος $R^t$ με την παρούσα μετάθεση $\pi^t$ που είναι στην ουσία η θέση του πρώτου στοιχείου του συνόλου $R^t$ στην μετάθεση $\pi^t$. \\
  
  \noindent Ξεκινάμε με μια ιστορική αναδρομή του προβλήματος παρουσιάζοντας τις διάφορες εκδοχές του προβλήματος που έχουν παρουσιαστεί στην βιβλιογραφία και τους αλγορίθμος που χρησιμοποιούνται για τον σχεδιασμό αλγορίθμων για την κάθε εκδοχή. Η παρουσίαση αυτών των τεχνικών είναι πολύ σημαντική καθώς κτίζουν μια καλή διαίσθηση για το πρόβλημα και επιπλέον βοηθάνε στην κατανόηση των αποτελεσμάτων της offline Dynamic εκδοχής με την οποία και ασχολούμαστε. \\
  
  \noindent Σε επόμενη φάση, συνεχίζουμε με μια ιστορική αναδρομή καθώς και παραλλαγές του προβλήματος. Ξεκινάμε με το πρόβλημα ανανέωσης λίστας(List-Update) και εξηγούμε πως χρησιμοποιώντας την μέθοδο δυναμικού που περιγράψαμε στην προηγούμενη ενότητα μπορούμε να λύσουμε το πρόβλημα με παράγοντα προσέγγισης 2. Επιπλέον, παρουσιάζουμε μια γενικότερη οικογένεια προβλημάτων που εξελίσσονται στον χρόνο και εξηγούμε πως μπορούμε να χρησιμοποιήσουμε Rounding γραμμικών προγραμμάτων προκειμένου να λάβουμε ακριβείς ή και προσεγγιστικές λύσεις για ορισμένα προβλήματα. Χρησιμοποιούμε το πρόβλημα Facility Reallocation on the Real Line για την παρουσίαση αυτής της τεχνικής. \\
  
 \noindent Στην τελευταία ενότητα παρουσιάζουμε τα αποτελέσματά μας για το πρόβλημα $\DSSC$. Ανάγοντας απ' το κλασικό πρόβλημα Κάλυψης Συνόλου(Set Cover), δείχνουμε αρχικά ότι το $\DSSC$ δεν μπορεί να προσεγγιστεί με παράγοντα προσέγγισης $O(1)$ εκτός και αν $\mathrm{P} = \mathrm{NP}$ καθώς και ότι οποιοσδήποτε αλγόριθμος με παράγοντα προσέγγισης $o( \log n)$ (αντίστοιχα $O(r)$) για το $\DSSC$ θα έδινε υπολογαριθμικό παράγοντα προσέγγισης για το Set Cover (ή $ο(r)$ αντίστοιχα αν θεωρήσουμε την εκδοχή όπου κάθε στοιχείο εμφανίζεται το πολύ $r$ φορές στα σύνολα κάλυψης). Η βασική μας συνεισφορά έγκειται στο οτι δείχνουμε πως το πρόβλημα $\DSSC$ μπορεί να προσεγγιστεί σε πολυωνυμικό χρόνο με παράγοντα $O( \log^2 n )$ στην γενική περίπτωση με randomized rounding και με παράγοντα $O(r^2)$, αν όλα τα σύνολα αιτημάτων κάλυψης έχουν πληθικότητα το πολύ $r$, με deterministic rounding.
\begin{keywordsgr}
  Κάλυμα συνόλου,
  Δυναμικοί Αλγόριθμοι,
  Αλγόριθμοι Προσέγγισης,
  Online Αλγόριθμοι,
  Combinatorial Optimization
\end{keywordsgr}

\end{abstractgr}

\begin{abstracten}%
We investigate the polynomial-time approximability of the multistage version of Min-Sum Set Cover ($\DSSC$), a natural and intriguing generalization of the classical List Update problem. In $\DSSC$, we maintain a sequence of permutations $(\pi^0, \pi^1, \ldots, \pi^T)$ on $n$ elements, based on a sequence of requests $\cR = (R^1, \ldots, R^T)$. We aim to minimize the total cost of updating $\pi^{t-1}$ to $\pi^{t}$, quantified by the Kendall tau distance $\dkt(\pi^{t-1}, \pi^t)$, plus the total cost of covering each request $R^t$ with the current permutation $\pi^t$, quantified by the position of the first element of $R^t$ in $\pi^t$. \\

\noindent We begin by examining the history of the problem and the main versions that have appeared in literature over time. Presenting the main techniques use for the design of algorithms is crucial since they build a nice intuition regarding the problem and pave the way to understand the results of the offline Dynamic version which is essentially the core of this thesis. \\

\noindent In the next chapter we continue with the history of the problem. By diving into the List Update problem we explain how the use of the Potential Method can be used to solve the problem with a 2-approximation algorithm. Furthermore, we present a general family of problems that are time evolving and show how we can use Rounding of appropriate Linear Programs to obtain exact or approximation algorithms for these problems. To demonstrate this technique we use the Facility Reallocation on the Real Line problem. \\

\noindent Using a reduction from Set Cover, we show that $\DSSC$ does not admit an $O(1)$-approximation, unless $\mathrm{P} = \mathrm{NP}$, and that any $o(\log n)$ (resp. $o(r)$) approximation to $\DSSC$ implies a sublogarithmic (resp. $o(r)$) approximation to Set Cover (resp. where each element appears at most $r$ times). Our main technical contribution is to show that $\DSSC$ can be approximated in polynomial-time within a factor of $O(\log^2 n)$ in general instances, by randomized rounding, and within a factor of $O(r^2)$, if all requests have cardinality at most $r$, by deterministic rounding. 

\begin{keywordsen}
  Set Cover,
  Dynamic Algorithms,
  Approximation Algorithms,
  Online Algorithms,
  Combinatorial Optimization
  
\end{keywordsen}

\end{abstracten}