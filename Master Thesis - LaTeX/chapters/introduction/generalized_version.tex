\section{The Generalized Min-Sum Set Cover}

After presenting the static version of Min Sum Set Cover we proceed with the generalization of the problem which was first introduced by Azar, Gamzu and Yin in \cite{AGY09}. In this problem we are once again given a universe $U$ with $|U| = n$, a collection of subsets of $U$, $S_1, S_2, ..., S_m \subseteq U$ and also an integer number for each set in the collection $K(S_i)$. Our purpose is to compute a permutation of $U$ in order to minimize the covering cost of each set: $\sum_{i=1}^{m} C( S_i )$ where $C(S_i)$ is the position of the $K(S_i)$-th element of $S_i$ that appears on the permutation. \\

In \cite{AGY09} the authors present an $O(logr)$-approximation algorithm where $r = max_{i} \{ |S_i| \}$. Subsequently, a constant factor approximation algorithm was given \cite{BGK10} based on Linear Programming. \\

In this section we present the main results of \cite{SW11} where the authors use a method called $\alpha$-point scheduling which helps them round the Linear Programming Relaxation of the problem and they obtain an algorithm that has a significantly better approximation ratio than \cite{BGK10}. The following results and techniques are rather important since several ideas we used for extracting our own results for $\DSSC$ are based on \cite{SW11}. \\

The first step is solving the following Linear Programming Relaxation of the problem. We define the variables $y_{i,t}$ which we want to be 1 if and only if $C(S_i) < t$ (that is, the $K(S_i)$-th element of the set $S_i$ appears before index t) and additionally the variables $x_{e,t}$ which we want to equal 1 if and only if the element e $e \in U$ appears at position $t$. Of course these hold for the integer program and since we solve a relaxation these variables will probably not be exactly equal to 1. Based on the constraints above we have the following linear program.

\begin{equation*}
    \begin{array}{ll@{}ll}
        \text min & \displaystyle \sum_{t = 1}^n \sum_{i = 1}^{m} ( 1 - y_{i,t} ) &\\
        \text{ s.t } & \displaystyle \sum_{e \in U} x_{e,t} = 1~~~~\text{για κάθε }t \leq n &&\\
        & \displaystyle \sum_{t=1}^{n} x_{e,t} = 1~~~~\text{για κάθε }e \in U &&\\
        & \displaystyle \sum_{e \in S \setminus A} \sum_{t'<t} x_{e,t} \geq (K(S_i) - |A|) \cdot y_{i,t}~~~~\text{ για κάθε } i \leq m, A \subseteq S_i, t \leq n &&\\
        & \displaystyle x_{e,t}, y_{i,t} \in [ 0, 1 ] ~~~~\text{ για κάθε } e \in U, i \leq m, t \leq n
    \end{array}
\end{equation*}

The Linear Program at first glance seems to have exponentially many constraints, since we create a constraint for every subset of every set $S_i$. It was shown in \cite{BGK10} that this problem can be dealt with and the exponentially many constraints can be separated in polynomial time so that we can solve the LP efficiently. \\

The basic algorithm for solving the problem with 28 appriximation ratio is based on $\alpha$ point scheduling. Specifically, for every $e \in U$ we select independently and uniformly $\alpha_e \in [0,1]$. Let $t_{e,\alpha_e}$ be the first time $t'$ s.t $\sum_{t=1}^{t'} x_{e,t} \geq \alpha_e$. We form the final permutation by ordering the elements $e \in U$ in decreasing order of $t_{e,\alpha_e}$ by breaking ties arbitrarily (but consistently). For more details on the analysis of the algorithm the interested reader may refer to \cite{BGK10}.